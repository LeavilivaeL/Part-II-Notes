\documentclass[10pt,a4paper]{article}
\usepackage[utf8]{inputenc}
\usepackage{amsmath}
\usepackage{amsfonts}
\usepackage{amssymb}
\usepackage{amsthm}
\usepackage{float}
\usepackage{mathtools}
\usepackage{geometry}[margin=1in]
\usepackage{xspace}
\usepackage{tikz}
\usepackage{mathrsfs}
\usetikzlibrary{shapes, arrows, decorations.pathmorphing}
\usepackage[parfill]{parskip}
\usepackage{subcaption}
\usepackage{stmaryrd}
\usepackage{marvosym}
\usepackage{dsfont}

\newcommand{\st}{\text{ s.t. }}
\newcommand{\contr}{\lightning}
\newcommand{\im}{\mathfrak{i}}
\newcommand{\R}{\mathbb{R}}
\newcommand{\Q}{\mathbb{Q}}
\newcommand{\C}{\mathbb{C}}
\newcommand{\F}{\mathbb{F}}
\newcommand{\K}{\mathbb{K}}
\newcommand{\N}{\mathbb{N}}
\newcommand{\Z}{\mathbb{Z}}
\renewcommand{\H}{\mathds{H}}
\newcommand{\nequiv}{\not\equiv}
\newcommand{\powset}{\mathcal{P}}
\renewcommand{\th}[1][th]{\textsuperscript{#1}\xspace}
\newcommand{\from}{\leftarrow}
\newcommand{\legendre}[2]{\left(\frac{#1}{#2}\right)}
\newcommand{\ow}{\text{otherwise}}
\newcommand{\imp}[2]{\underline{\textit{#1.}$\implies$\textit{#2.}}}
\let\oldexists\exists
\renewcommand{\exists}{\oldexists\;}
\renewcommand{\hat}{\widehat}
\renewcommand{\tilde}{\widetilde}
\newcommand{\one}{\mathds{1}}
\newcommand{\under}{\backslash}
\newcommand{\injection}{\hookrightarrow}
\newcommand{\surjection}{\twoheadrightarrow}
\newcommand{\jacobi}{\legendre}
\newcommand{\floor}[1]{\lfloor #1 \rfloor}
\newcommand{\ceil}[1]{\lceil #1 \rceil}
\newcommand{\cbrt}[1]{\sqrt[3]{#1}}

\DeclareMathOperator{\ex}{ex}
\DeclareMathOperator{\id}{id}
\DeclareMathOperator{\upper}{Upper}
\DeclareMathOperator{\dom}{dom}

\DeclareMathOperator{\charr}{char}
\DeclareMathOperator{\Image}{im}
\DeclareMathOperator{\ord}{ord}
\DeclareMathOperator{\lcm}{lcm}
\let\emph\relax
\DeclareTextFontCommand{\emph}{\bfseries\em}

\newtheorem{theorem}{Theorem}[section]
\newtheorem{lemma}[theorem]{Lemma}
\newtheorem{corollary}[theorem]{Corollary}
\newtheorem{proposition}[theorem]{Proposition}
\newtheorem{conjecture}[theorem]{Conjecture}

\tikzset{sketch/.style={decorate,
 decoration={random steps, amplitude=1pt, segment length=5pt}, 
 line join=round, draw=black!80, very thick, fill=#1
}}

\title{Sogic and Let Theory}
\begin{document}
\maketitle

\section{Propositional Logic}
Let $P$ be a set of \emph{primitive propositions}, i.e. $P$ is a set of symbols with $(, ), \bot, \implies \notin P$. Unless stated otherwise (i.e. that $P$ is uncountable), we may assume that $P = \{p_1, p_2, \ldots \}$.

The set of \emph{propositions}, denoted by $L(P)$ or simply just $L$, is defined inductively as follows:
\begin{enumerate}
\item $P \subset L$
\item $\bot \in L$, called \false
\item if $p, q \in L$, then $(p \implies q) \in L$
\end{enumerate}

Each proposition is a string of symbols from $P \cup \{(,),\bot,\implies\}$, for instance we have the propositions $p_1, (p_1 \implies p_1), ((p_1\implies p_2)\implies(p_2 \implies(\bot \implies p_3)))$. For readability, we often draw symbols $(,)$ in different ways, for instance as $[, (, \Big($.

Sometimes we omit the outside pair of parentheses when writing down propositions, for instance $p_1 \implies p_2$ is shorthand for $(p_1 \implies p_2)$.

Also we use some abbreviations, e.g.:
\begin{itemize}
\item[\textsc{Not:}] $\neg p$ to mean $(p \implies \bot)$
\item[\textsc{Or:}] $p \lor q$ to mean $(\neg p \implies q)$
\item[\textsc{And:}] $p \land q$ to mean $\neg(\neg p \lor \neg q)$
\end{itemize}

What do we mean by $L$ ``defined inductively"? Define $L_0 = P \cup \{\bot\}$. Then, given $L_n$, we can define $L_{n+1} = L_{n-1} \cup \{(p\implies q) : p, q \in L_{n-1}\}$. Then we set $L = \bigcup_{n=0}^{\infty} L_n$. Note: if $p \in L \setminus(P \cup \{\bot\})$, then it is easy to show that there are \emph{unique} $q, r \in L$ with $p = (q \implies r)$.

\subsection{Semantic Entailment}
A \emph{valuation} is a function $v: L \to \{0,1\}$ satisfying:
\begin{enumerate}
\item $v(\bot) = 0$
\item For all $p, q \in L, v(p\implies q) = \begin{cases} 0 & v(p) = 1, v(q) = 0 \\ 1 & \ow \end{cases}$.
\end{enumerate}
If $p \in L$ and $v(p) = 1$ for every valuation, we say that $p$ is a \emph{tautology}, and write $\taut p$.

\underline{Examples:}
\begin{enumerate}
\item $\taut (p \implies p)$

\begin{tabular}{c|c}
$v(p)$ & $v(p\implies p)$ \\\hline
0 & 1 \\
1 & 1
\end{tabular}

So this is a tautology.

\item $\taut (p \implies (q \implies p))$

\begin{tabular}{c|c|c|c}
$p$ & $q$ & $q \implies p$ & $p \implies (q \implies p)$\\\hline
0 & 0 & 1 & 1\\
0 & 1 & 0 & 1\\
1 & 0 & 1 & 1\\
1 & 1 & 1 & 1
\end{tabular}

So this is a tautology.

\item Is $\taut (p \implies (q \implies r)) \implies ((p\implies q)\implies(p \implies r))$?

Suppose not. Then for some $p,q,r$ and valuation $v$ we have:

\hspace*{1cm}$v(p \implies (q\implies r)) = 1$\\
\hspace*{1cm}$v((p\implies q)\implies(p \implies r)) = 0$.

So $v(p \implies q) = 1, v(p \implies r) = 0$. Hence $v(p) = 1, v(r) = 0, v(q) = 1$. But then $v(q \implies r) = 0$, and so $v(p \implies(q\implies r)) = 0 \contr$.

\item $\taut ((p \implies \bot) \implies \bot) \implies p$, i.e. $ \neg \neg p \implies p$, i.e. $(\neg p \lor p)$. This is the Law of the Excluded Middle, and is also a tautology.
\end{enumerate}
\end{document}