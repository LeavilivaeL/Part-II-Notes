\documentclass[10pt,a4paper]{article}
\usepackage[utf8]{inputenc}
\usepackage{amsmath}
\usepackage{amsfonts}
\usepackage{amssymb}
\usepackage{amsthm}
\usepackage{float}
\usepackage{mathtools}
\usepackage{geometry}[margin=1in]
\usepackage{xspace}
\usepackage{tikz}
\usepackage{mathrsfs}
\usetikzlibrary{shapes, arrows, decorations.pathmorphing, ducks, automata}
\usepackage[parfill]{parskip}
\usepackage{subcaption}
\usepackage{stmaryrd}
\usepackage{marvosym}
\usepackage{dsfont}
\usepackage{centernot}

\setlength{\parindent}{1em}

\newcommand{\st}{\text{ s.t. }}
\newcommand{\contr}{\lightning}
\newcommand{\im}{\mathfrak{i}}
\newcommand{\R}{\mathbb{R}}
\newcommand{\Q}{\mathbb{Q}}
\newcommand{\C}{\mathbb{C}}
\newcommand{\F}{\mathbb{F}}
\newcommand{\K}{\mathbb{K}}
\newcommand{\N}{\mathbb{N}}
\newcommand{\Z}{\mathbb{Z}}
\newcommand{\E}{\mathbb{E}}
\renewcommand{\P}{\mathbb{P}}
\renewcommand{\H}{\mathds{H}}
\newcommand{\nequiv}{\not\equiv}
\newcommand{\powset}{\mathcal{P}}
\renewcommand{\th}[1][th]{\textsuperscript{#1}\xspace}
\newcommand{\from}{\leftarrow}
\newcommand{\legendre}[2]{\left(\frac{#1}{#2}\right)}
\newcommand{\ow}{\text{otherwise}}
\newcommand{\imp}[2]{\underline{\textit{#1.}$\implies$\textit{#2.}}}
\let\oldexists\exists
\renewcommand{\exists}{\oldexists\;}
\renewcommand{\hat}{\widehat}
\renewcommand{\tilde}{\widetilde}
\newcommand{\one}{\mathds{1}}
\newcommand{\under}{\backslash}
\newcommand{\injection}{\hookrightarrow}
\newcommand{\surjection}{\twoheadrightarrow}
\newcommand{\jacobi}{\legendre}
\newcommand{\floor}[1]{\lfloor #1 \rfloor}
\newcommand{\ceil}[1]{\lceil #1 \rceil}
\newcommand{\cbrt}[1]{\sqrt[3]{#1}}
\renewcommand{\angle}[1]{\langle #1 \rangle}
\renewcommand{\o}{\mathfrak{o}}
\newcommand{\dbangle}[1]{\angle{\angle{#1}}}
\newcommand{\false}{\textsc{False}}
\newcommand{\taut}{\vDash}
\newcommand{\ket}[1]{|#1\rangle}
\newcommand{\bra}[1]{\langle #1|}
\newcommand{\braket}[2]{\langle #1 | #2 \rangle}
\newcommand{\colvec}[1]{\begin{pmatrix} #1 \end{pmatrix}}

\DeclareMathOperator{\ex}{ex}
\DeclareMathOperator{\id}{id}
\DeclareMathOperator{\upper}{Upper}
\DeclareMathOperator{\dom}{dom}
\DeclareMathOperator{\disc}{disc}
\DeclareMathOperator{\charr}{char}
\DeclareMathOperator{\Image}{im}
\DeclareMathOperator{\ord}{ord}
\DeclareMathOperator{\lcm}{lcm}
\DeclareMathOperator{\aut}{Aut}
\DeclareMathOperator{\diag}{diag}
\DeclareMathOperator{\stab}{stab}
\DeclareMathOperator{\trace}{trace}
\DeclareMathOperator{\ecl}{ecl}
\DeclareMathOperator{\Span}{Span}
\DeclareMathOperator{\Gal}{Gal}
\DeclareMathOperator{\Var}{Var}
\let\Re\relax
\let\Im\relax
\DeclareMathOperator{\Re}{\mathfrak{Re}}
\DeclareMathOperator{\Im}{\mathfrak{Im}}
\DeclareMathOperator{\Frac}{Frac}

\let\emph\relax
\DeclareTextFontCommand{\emph}{\bfseries\em}

\newtheorem{theorem}{Theorem}[section]
\newtheorem{lemma}[theorem]{Lemma}
\newtheorem{corollary}[theorem]{Corollary}
\newtheorem{proposition}[theorem]{Proposition}
\newtheorem{conjecture}[theorem]{Conjecture}

\tikzset{sketch/.style={decorate,
 decoration={random steps, amplitude=1pt, segment length=5pt}, 
 line join=round, draw=black!80, very thick, fill=#1
}}

\title{Algebraic Geometry}
\begin{document}
\maketitle

\setcounter{section}{-1}

\section{Introduction}
What is algebraic geometry? Broadly speaking, it is the study of the geometry of solutions to systems of polynomial equations. For example, in $\R^2$, if we have the set $X$ of solutions to $\{(x,y)\in \R^2 : x^2+y^2 = 1\}$, then we know that this set forms a circle, and we know lots of geometric facts about circles. If we take a more complicated function, such as $y^2 = x^3-x$, we get something that looks like:
\begin{tikzpicture}% fill this in

\end{tikzpicture}

If we instead think about complex solutions, we get something of the form of a torus minus a single point, with another rich geometric structure.

In $\C^3$, if $X = \{(x,y,z) \in \C^3 : x^3 + y^3 + z^3 = 1\}$, then $X$ contains 27 lines: $x = -\xi^m y, z = \xi^n$ for $i,j \in \{0,1,2\}$ gives 9 of them, and the other 18 come by rotating $x,y,z$ in this linear system.

In $\R^3$, consider the equation $1+x^3+y^3+z^3 = (1+x+y+z)^3$. %find pic of this

\section{Basic Setup}
Fix a field $K$. We define an \emph{affine n-space over K} to be $\mathbb{A}^n \coloneqq K^n$. Let $A \coloneqq K[x_1,x_2,\ldots,x_n]$ be the polynomial ring in $n$ variables over $K$, and let $S \subseteq A$ be a subset of $A$. We then define $Z(S)$, the \emph{zero set of S} to be the set of all $n$-tuples $(a_1,\ldots,a_n) \in \mathbb{A}^n$ where $f(a_1,\ldots,a_n) = 0$ for all $f \in S$.

\begin{proposition}
\item
\begin{enumerate}
\item $Z(\{0\}) = \mathbb{A}^n$
\item $Z(A) = \emptyset$
\item $Z(S_1\cdot S_2) = Z(S_1) \cup Z(S_2)$, where $S_1\cdot S_2 = \{f_1\cdot f_2 : f_1 \in S_1, f_2 \in S_2\}$.
\item Let $I$ be an index set, $S_i \subseteq A$ for each $i \in I$. Then $\bigcap_{i \in I} Z(S_i) = Z(\bigcup_{i \in I} S_i)$
\end{enumerate}
\end{proposition}
\begin{proof} \textit{1., 2.} are obvious
\begin{enumerate}
\item If $p \in Z(S_1) \cup Z(S_2)$, then either $p \in Z(S_1)$ or $p \in Z(S_2)$. If $p \in Z(S_1)$, then $f_1(p) = 0$ for all $f_1 \in S_1$, and so $f_1(p)\cdot f_2(p) = 0$ for all $f_1 \in S_1, f_2 \in S_2$, so $p \in Z(S_1\cdot S_2)$, and similarly for if $p \in Z(S_2)$.

Conversely, suppose that $p \in Z(S_1\cdot S_2)$, and $p \notin Z(S_1)$. Then there is some $f_1 \in S_1$ with $f_1(p) \neq 0$. But $f_1(p)\cdot f_2(p) = 0$ for all $f_2 \in S_2$, and so $f_2(p) = 0$ for all $f_2 \in S_2)$, so $p \in Z(S_2)$.

\item If $p \in Z(S_i)$ for all $i \in I$, then $f_i(p) = 0$ for all $f_i \in S_i$, and so for all $f \in \bigcup_{i} S_i$, so $p \in Z(\bigcup_{i \in I}S_i)$.

Conversely, if $p \in Z(\bigcup_i S_i)$, then $f(p) = 0$ for all the polynomials in $\bigcup_i S_i$, and so $p \in \bigcap_i S_i$.
\end{enumerate}
\end{proof}
These four properties should remind you of the four axioms for a topology.

A subset of $\mathbb{A}^n$ is \emph{algebraic} if it of the form $Z(S)$ for some $S\subseteq A$. A \emph{Zariski open set} in $\mathbb{A}^n$ is a set of the form $\mathbb{A}^n \setminus Z(S)$ for some $S \subseteq \mathbb{A}^n$. This proposition tells us that the Zariski open sets define a topology on $\mathbb{A}^n$, called the \emph{Zariski topology}.

\underline{Examples:}
\begin{enumerate}
\item $K = \C$. The Zariski open (or closed) subsets of $\C^n = \mathbb{A}^n$ are in particular open (or closed) in the usual Euclidean sense, but not vice versa.
\item For any $K$, consider $\mathbb{A}^1, A = K[x], S \subseteq K[x]$. If $S$ has a non-zero element, then $Z(S)$ is finite. Thus the closed sets are the finite subsets of $\mathbb{A}^1$, and all of $\mathbb{A}^1$. The open sets are $\emptyset$ and all the co-finite sets (i.e. sets with finite complement).
\end{enumerate}

Recall that, if $A$ is any commutative ring with $S \subseteq A$ a subset, then the \emph{ideal generated by S} is the ideal $A \supseteq \angle{S} = \{\sum_{i=1}^q f_ig_i : q \geq 0, f_i \in S, g_i \in A\}$, or the smallest ideal of $A$ containing $S$.

\begin{lemma}
Let $S \subseteq A = K[x_1, \ldots, x_n]$. Then $Z(S) = Z(\angle{S})$.
\end{lemma}
\begin{proof}
If $p \in Z(S)$, then for $f_1, \ldots, f_q \in S; g_1, \ldots, g_q \in A$ we have:
\begin{align*}
\left(\sum_{i=1}^q f_ig_i\right)(p) = \sum_{i=1}^q f_i(p)g_i(p) = \sum_{i=1}^p 0\cdot g_i(p) = 0
\end{align*}
So $p \in Z(\angle{S})$, and so $Z(S)\subseteq Z(\angle{S})$. 

\hspace*{-1em}The other inclusion follows from the fact that $S \subseteq \angle{S}$, we must have $Z(\angle{S}) \subseteq Z(S)$.
\end{proof}

Let $X \subseteq \mathbb{A}^n$ be a subset. Define $I(X) \coloneqq \{f \in A : f(p) = 0 \;\forall p \in X\}$, the \emph{ideal of X}. Note that $I(X)$ is indeed an ideal, since if $f, g \in I(X)$ then $f+g \in I(X)$, and if $f \in I(X), g \in A$, then $f\cdot g \in I(X)$.

Note that if $S_1\subseteq S_2 \subseteq A_1$, then $Z(S_2) \subseteq Z(S_1)$, and if $X_1 \subseteq X_2$, then $I(X_2) \subseteq I(X_1)$.
\end{document}
