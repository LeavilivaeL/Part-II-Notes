\documentclass[10pt,a4paper]{article}
\usepackage[utf8]{inputenc}
\usepackage{amsmath}
\usepackage{amsfonts}
\usepackage{amssymb}
\usepackage{amsthm}
\usepackage{float}
\usepackage{mathtools}
\usepackage{geometry}[margin=1in]
\usepackage{xspace}
\usepackage{tikz}
\usepackage{mathrsfs}
\usetikzlibrary{shapes, arrows, decorations.pathmorphing, ducks, automata}
\usepackage[parfill]{parskip}
\usepackage{subcaption}
\usepackage{stmaryrd}
\usepackage{marvosym}
\usepackage{dsfont}
\usepackage{centernot}

\setlength{\parindent}{1em}

\newcommand{\st}{\text{ s.t. }}
\newcommand{\contr}{\lightning}
\newcommand{\im}{\mathfrak{i}}
\newcommand{\R}{\mathbb{R}}
\newcommand{\Q}{\mathbb{Q}}
\newcommand{\C}{\mathbb{C}}
\newcommand{\F}{\mathbb{F}}
\newcommand{\K}{\mathbb{K}}
\newcommand{\N}{\mathbb{N}}
\newcommand{\Z}{\mathbb{Z}}
\newcommand{\E}{\mathbb{E}}
\renewcommand{\P}{\mathbb{P}}
\renewcommand{\H}{\mathds{H}}
\newcommand{\nequiv}{\not\equiv}
\newcommand{\powset}{\mathcal{P}}
\renewcommand{\th}[1][th]{\textsuperscript{#1}\xspace}
\newcommand{\from}{\leftarrow}
\newcommand{\legendre}[2]{\left(\frac{#1}{#2}\right)}
\newcommand{\ow}{\text{otherwise}}
\newcommand{\imp}[2]{\underline{\textit{#1.}$\implies$\textit{#2.}}}
\let\oldexists\exists
\renewcommand{\exists}{\oldexists\;}
\renewcommand{\hat}{\widehat}
\renewcommand{\tilde}{\widetilde}
\newcommand{\one}{\mathds{1}}
\newcommand{\under}{\backslash}
\newcommand{\injection}{\hookrightarrow}
\newcommand{\surjection}{\twoheadrightarrow}
\newcommand{\jacobi}{\legendre}
\newcommand{\floor}[1]{\lfloor #1 \rfloor}
\newcommand{\ceil}[1]{\lceil #1 \rceil}
\newcommand{\cbrt}[1]{\sqrt[3]{#1}}
\renewcommand{\angle}[1]{\langle #1 \rangle}
\renewcommand{\o}{\mathfrak{o}}
\newcommand{\dbangle}[1]{\angle{\angle{#1}}}
\newcommand{\false}{\textsc{False}}
\newcommand{\taut}{\vDash}
\newcommand{\ket}[1]{|#1\rangle}
\newcommand{\bra}[1]{\langle #1|}
\newcommand{\braket}[2]{\langle #1 | #2 \rangle}
\newcommand{\colvec}[1]{\begin{pmatrix} #1 \end{pmatrix}}

\DeclareMathOperator{\ex}{ex}
\DeclareMathOperator{\id}{id}
\DeclareMathOperator{\upper}{Upper}
\DeclareMathOperator{\dom}{dom}
\DeclareMathOperator{\disc}{disc}
\DeclareMathOperator{\charr}{char}
\DeclareMathOperator{\Image}{im}
\DeclareMathOperator{\ord}{ord}
\DeclareMathOperator{\lcm}{lcm}
\DeclareMathOperator{\aut}{Aut}
\DeclareMathOperator{\diag}{diag}
\DeclareMathOperator{\stab}{stab}
\DeclareMathOperator{\trace}{trace}
\DeclareMathOperator{\ecl}{ecl}
\DeclareMathOperator{\Span}{Span}
\DeclareMathOperator{\Gal}{Gal}
\DeclareMathOperator{\Var}{Var}
\let\Re\relax
\let\Im\relax
\DeclareMathOperator{\Re}{\mathfrak{Re}}
\DeclareMathOperator{\Im}{\mathfrak{Im}}
\DeclareMathOperator{\Frac}{Frac}

\let\emph\relax
\DeclareTextFontCommand{\emph}{\bfseries\em}

\newtheorem{theorem}{Theorem}[section]
\newtheorem{lemma}[theorem]{Lemma}
\newtheorem{corollary}[theorem]{Corollary}
\newtheorem{proposition}[theorem]{Proposition}
\newtheorem{conjecture}[theorem]{Conjecture}

\tikzset{sketch/.style={decorate,
 decoration={random steps, amplitude=1pt, segment length=5pt}, 
 line join=round, draw=black!80, very thick, fill=#1
}}

\title{Number Fields}
\begin{document}
\maketitle

\section{Algebraic Numbers and Algebraic Integers; Number Fields}
Here, we will use $F$ to denote any field containing $\Q$, for instance $F = \C$. Recall that an element $\alpha \in F$ is \emph{algebraic} (over $\Q$) if it is the root of some polynomial in $\Q[x]$. If so, there is a unique monic polynomial $m_{\alpha} \in \Q[x]$ of minimal degree with $m_{\alpha}(\alpha) = 0$, called the \emph{minimal polynomial} of $\alpha$. The \emph{degree} of $\alpha$ is the degree of $m_{\alpha}$

\begin{proposition}
Suppose $\alpha \in F$ is algebraic. Then $m_{\alpha}$ is irreducible in $\Q[x]$, and if $f \in \Q[x]$, then $f(\alpha) = 0 \iff m_{\alpha} | f$.
\end{proposition}
\begin{proof}
If $m_{\alpha} = fg$, then $f(\alpha)g(\alpha) = 0$, and since fields are integral domains we have $f(\alpha) = 0$ or $g(\alpha) = 0$. By minimality of degree, $f$ or $g$ is constant.

If $f(\alpha) = 0$, we write $f = gm_{\alpha} + h$, with $g, h \in \Q[x]$, and $\deg h < \deg m_{\alpha}$. Then $h(\alpha) = f(\alpha) - g(\alpha)m_{\alpha}(\alpha) = 0$, and so by minimality $h = 0$ and $m_{\alpha}|f$.

I.e. $\{f : f(\alpha) = 0\}$ is a principal ideal in $\Q[x]$ generated by $m_{\alpha}$
\end{proof}

If $\alpha \in F$, define $\Q(\alpha)$ to be the smallest subfield of $F$ containing $\alpha$. Explicitly, it can be shown that $\Q(\alpha) = \left\{ \frac{f(\alpha)}{g(\alpha)} : f, g \in \Q[x], g(\alpha) \neq 0\right\}$.

\begin{proposition}
If $\alpha \in F$ is algebraic of degree $n$, then $1, \alpha, \ldots, \alpha^{n-1}$ is a $\Q$-basis for $\Q(\alpha)$. Conversely, if $[\Q(\alpha: \Q] \coloneqq \dim_{\Q} \Q(\alpha)$ is finite, say $n$, then $\alpha$ is algebraic of degree $n$.
\end{proposition}
\begin{proof}
Consider the homomorphism $\phi:\Q[x] \to F; f\mapsto f(\alpha)$. Then $\ker(\phi) = (m_\alpha)$ which is maximal, so $\Image \phi$ is a field, and hence equal to $\Q(\alpha)$. As $\deg m_{\alpha} = n$, a basis for $\Q[x]/(m_\alpha)$ is $1, x, \ldots, x^{n-1}$, and hence $1, \alpha, \ldots, \alpha^{n-1}$ is a basis for $\Q(\alpha)$.

For the converse part, if $[\Q(\alpha):\Q] = n < \infty$, then $1, \alpha, \ldots, \alpha^n$ are linearly dependent and so $\alpha$ is algebraic of some degree. By the first part, this degree is $n$.
\end{proof}

\begin{proposition}
$\{\alpha \in F : \alpha$ algebraic$\}$ is a subfield of $F$.
\end{proposition}
\begin{proof}[Galois theory]
It is enough to prove that it is closed under $+, \times$ and inverse. For $+$ and $\times$ see \textbf{1.6} below for a stronger statement. If $0 \neq \alpha$ is algebraic, then $\sum^n b_j \alpha^j = 0 \implies \sum^n b_{n-j}(\alpha^{-1})^j = 0$, and so $\alpha^{-1}$ is algebraic.
\end{proof}

$\alpha \in F$ is an \emph{algebraic integer} if there exists a monic polynomial $f \in \Z[x]$ with $f(\alpha) = 0$.
\stepcounter{theorem}

\begin{lemma}
\item
\begin{enumerate}
\item Let $\alpha \in F$. Then the following are equivalent:
\begin{enumerate}
\item $\alpha$ is an algebraic integer
\item $\alpha$ is algebraic and $m_\alpha \in \Z[x]$
\item $\Z[\alpha]$ is a finitely generated $\Z$-module
\end{enumerate}
If these hold, then $1, \alpha, \ldots, \alpha^{d-1}$ is a $\Z$-bases for $\Z[\alpha]$, with $d = \deg \alpha$.
\item $\alpha \in \Q$ is an algebraic integer $\iff \alpha \in \Z$
\end{enumerate}
\end{lemma}
Recall the notation that, if $\alpha_1, \ldots, \alpha_n \in F$, then $\Z[\alpha_1, \ldots, \alpha_n]$ is the smallest subring of $F$ containing $\{\alpha_i:i\in[n]\}$, i.e. the set of all finite sums of terms of the form $A\alpha_1^{i_1}\ldots\alpha_n^{i_n}$ for $A, i_1, \ldots, i_n \in \Z$.

\begin{proof}\item
\begin{enumerate}
\item
\begin{enumerate}
\imp{a}{b} Suppose $f(\alpha) = 0, f \in \Z[x]$, $f$ monic. Then \textbf{1.1} gives that $f = gm_{\alpha}$ for some $g \in \Q[x]$ necessarily monic. Gauss's lemma from GRM gives us that $m_{\alpha}, g$ are in $\Z[x]$.

\imp{b}{c} Write $m_{\alpha} = x^d + \sum_{j=1}^{d-1} b_j x^j$, for $b_j \in \Z$. Then $\alpha^d = -\sum_{j=1}^{d-1} b_j \alpha^j$, from which we say that every $\alpha^n$ is a $\Z$-linear combination of $1, \alpha, \ldots, \alpha^{d-1}$. So $\Z[\alpha]$ is generated by $1, \alpha, \ldots, \alpha^{d-1}$ as a $\Z$-module. There is no linear relation between $1, \alpha, \ldots, \alpha^{d-1}$, as $d = \deg \alpha$. So $\Z[\alpha]$ is finitely generated and $1, \alpha, \ldots, \alpha^{d-1}$ is a $\Z$-basis.

\imp{c}{a} Assume $\Z[\alpha]$ is finitely generated by $g_1(\alpha), \ldots, g_r(\alpha)$. For some $g_i \in \Z[x]$. Let $k = \max\{\deg g_i\}$. Then $\Z[\alpha]$ is certainly generated by $1, \alpha, \ldots, \alpha^k$ as a $\Z$-module. So $\alpha^{k+1} = \sum_{j=0}^k b_j \alpha^j$ for $b_j \in \Z$, and so $\alpha$ is an algebraic integer.
\end{enumerate}
\item $\alpha \in \Q \implies m_{\alpha} = x-\alpha$, and so $\alpha$ is an algebraic integer $\iff \alpha \in \Z$ using $(a) \iff (b)$.
\end{enumerate}
\end{proof}

\begin{theorem}
If $\alpha, \beta \in F$ are algebraic integers, then so are $\alpha\beta, \alpha\pm\beta$.
\end{theorem}
\begin{proof}
The $\Z$-module $\Z[\alpha,\beta]$ is generated by $\{\alpha^i\beta^j : 0 \leq i < \deg \alpha; 0 \leq j < \deg \beta\}$, and so is finitely generated. Hence so is the submodule $\Z[\alpha\beta]\subseteq \Z[\alpha,\beta]$. So $\alpha\beta$ is an algebraic integer by \textbf{1.4}. The same applies for $\alpha+\beta, \alpha-\beta$.
\end{proof}

Now to introduce the main characters of this course:

An \emph{algebraic number field} (or just \emph{number field}) is a field $K \supset \Q$ which is a finite extension, i.e. $[K:\Q] < \infty$. The \emph{ring of integers of K}, written $\o_K$, is the set of algebraic integers in $K$. By \textbf{1.6} it is a ring. It is useful to have the converse:

\begin{theorem}[Primitive Element]
If $K$ is a number field, then $K = \Q(\alpha)$ for some $\alpha \in K$.
\end{theorem}
\begin{proof}
Done in Galois theory.
\end{proof}
\end{document}